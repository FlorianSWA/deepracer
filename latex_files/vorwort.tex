\chapter{Preface}

%\noindent
%Übrigens, hier im Vorwort kann man kurz auf die Entstehung  des Dokuments eingehen.
%Hier ist auch der Platz für allfällige Danksagungen (\zB an den Betreuer, 
%den Begutachter, die Familie, den Hund, ...), Widmungen und philosophische 
%Anmerkungen. Das sollte man allerdings auch nicht übertreiben und sich auf 
%einen Umfang von maximal zwei Seiten beschränken.

This is \textbf{Version \htldiplDate} of our diploma thesis created in LaTex.
During the creation of this document we were able to take a glimpse at the immense possibilities of machine learning. Artificial intelligence and robotics are, as seen over the last years, topics of immense importance, more so now than ever. This rise in importance and usage creates the necessity to be informed about state of the art technology and have some form of basic understanding of the principles applied in these technologies. Artificial intelligence is nothing new and has been used by our robotics teams, primarily for image and pattern recognition. The lack of proper self-driving vehicles was mainly due to the technological and financial requirements. 
