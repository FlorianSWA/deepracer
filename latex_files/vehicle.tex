\chapter{Vehicle}

The DeepRacer itself represents a core part of this thesis, as it is used to test the trained models in a real-world environment. Our school acquired two of these cars. one of which we are using for this thesis. Below is a list of all parts included with one DeepRacer vehicle.

\begin{enumerate}
    \item Vehicle chassis
    \item Vehicle body cover
    \item Compute module battery
    \item Power cable and power adapter
    \item Vehicle battery
\end{enumerate}

\section{Chassis and Accessories}
The vehicle itself consists primarily of a four wheel drive chassis which holds all other components. The chassis can be further separated into a lower part, which contains the brushed electric motor, and an upper part that carries the compute module and a power bank to supply it \footcite{AWS19}. The entire car is build on a scale of 1/18 to a real car, meaning proportions like distance between wheels were kept realistic. This is especially apparent while driving, as one would expect better manoeuvrability and smaller turning radius.

At the front there are three USB ports used to mount the camera and other equipment like keyboards. As a crucial part of any self-driving car, the camera provided with the DeepRacer provides a 4-megapixel image directly to the compute module. Since we are working with the first edition of the DeepRacer vehicle and not with the newer DeepRacer Evo, which has stereo cameras and a LiDAR\footnote{Light detection and ranging} sensor, object avoidance and head-to-head racing are not supported by default. It is however possible to purchase an upgrade kit for 249,00 US\$, which includes an additional stereo camera and the LiDAR sensor.

The car uses Ackermann front-wheel steering to turn the wheels on the inside of a turn. This means that the left and right front wheels generally turn at different angles.\footcite{AWS19} The centered steering angle as well as the maximum steering angles can be configured via the web interface. This is necessary so that the car can dive in a straight line. Incorrectly configured steering can lead to severe hits  in performance as the model has to compensate for the drift caused by the misaligned wheels. 

\section{Compute Module}
The upper component houses the ``brain'' of our car. The compute module consists of an Intel Atom
%todo: Fix trademark symbol
processor, 4 Gigabyte of RAM and 32 Gigabyte of memory, which can be expanded upon via a SD card. This hardware is running Linux Ubuntu 16.04.3 with Intel OpenVINO\texttrademark{} and ROS\footnote{Robot Operating System} Kinetic. Apart from that the chassis offers 4 USB type A ports, 1 USB type C port, 1 Micro-USB port and one HDMI port. The USB type C port is used to supply the compute module with power, while the HDMI port offers the ability to connect a display and directly access the modules operating system.

In addition to that, the compute module hosts its own web interface, from which the vehicle can be monitored, remotely controlled and configured. This interface is meant to be the main access point when working with the car. Its most important task is loading new models onto the car. Besides that, it is possible to reconfigure the vehicles steering angles. \textbf{As of writing this} we were not able to correctly configure the steering angles so that both front wheels are aligned correctly.
% maybe insert image of DeepRacer console here