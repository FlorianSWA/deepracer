\chapter{Einleitung}
\label{cha:Einleitung}

\section{Zielsetzung}
Dieses Dokument ist als vorwiegend technische Starthilfe für das
Erstellen einer Diplomarbeit mit \latex
gedacht und ist die Weiterentwicklung einer früheren
Vorlage\footnote{Nicht mehr verfügbar.} für das Arbeiten mit
Microsoft \emph{Word}. Während ursprünglich daran gedacht war, die
bestehende Vorlage einfach in \latex zu übernehmen, wurde rasch
klar, dass allein aufgrund der großen Unterschiede zum Arbeiten
mit \emph{Word} ein gänzlich anderer Ansatz notwendig wurde. Dazu
kamen zahlreiche Erfahrungen mit Diplomarbeiten in den
nachfolgenden Jahren, die zu einigen zusätzlichen Hinweisen Anlass gaben.

Das vorliegende Dokument dient einem zweifachen Zweck: 
\emph{erstens} als Erläuterung und Anleitung, \emph{zweitens} als
direkter Ausgangspunkt für die eigene Arbeit. Angenommen wird,
dass der Leser bereits über elementare Kenntnisse im Umgang mit
\latex verfügt. In diesem Fall sollte -- eine einwandfreie
Installation der Software vorausgesetzt -- der Arbeit nichts mehr
im Wege stehen. Auch sonst ist der Start mit \latex\ nicht
schwierig, da viele hilfreiche Informationen auf den zugehörigen
Webseiten zu finden sind (s.\ Abschn.~\ref{sec:latex}).





\section{Warum {\latex}?}

Diplomarbeiten, Dissertationen und Bücher im
technisch-natur\-wissen\-schaft\-lichen Bereich werden
traditionell mithilfe des Textverarbeitungssystems \latex
\footcite{Lamport94,Lamport95} gesetzt. Das hat gute Gründe, denn
\latex ist bzgl.\ der Qualität des Druckbilds, des Umgangs mit
mathematischen Elementen, Literaturverzeichnissen etc.\
unübertroffen und ist noch dazu frei verfügbar. Wer mit \latex
bereits vertraut ist, sollte es auch für die Diplomarbeit
unbedingt in Betracht ziehen, aber auch für den Anfänger sollte
sich die zusätzliche Mühe am Ende durchaus lohnen.

Für den professionellen elektronischen Buchsatz wurde bisher
häufig \emph{Adobe Framemaker} verwendet, allerdings ist diese
Software teuer und komplex. Eine modernere Alternative dazu ist
\emph{Adobe InDesign}, wobei allerdings die Erstellung
mathematischer Elemente und die Verwaltung von Literaturverweisen
zur Zeit nur rudimentär unterstützt werden.%
\footnote{Angeblich werden aber für den (sehr sauberen) Schriftsatz 
in \emph{InDesign} ähnliche Algorithmen wie in \latex\ verwendet.}

Microsoft \emph{Word} gilt im Unterschied zu \latex, 
\emph{Framemaker} und \emph{InDesign} übrigens nicht als professionelle
Textverarbeitungssoftware, obwohl es immer häufiger auch von
großen Verlagen verwendet wird.%
\footnote{Siehe auch \url{http://latex.tugraz.at/mythen.php}.}
Das Schriftbild in \emph{Word}
lässt -- zumindest für das geschulte Auge -- einiges zu wünschen
übrig und das Erstellen von Büchern und ähnlich großen Dokumenten
wird nur unzureichend unterstützt. Allerdings ist \emph{Word} sehr
verbreitet, flexibel und vielen Benutzern zumindest oberflächlich
vertraut, sodass das Erlernen eines speziellen Werkzeugs wie
\latex\ ausschließlich für das Erstellen einer Diplomarbeit
manchen verständlicherweise zu mühevoll ist. Man sollte es daher
niemandem übel nehmen, wenn er/sie sich auch bei der Diplomarbeit
auf \emph{Word} verlässt. Im Endeffekt lässt sich mit etwas
Sorgfalt (und ein paar Tricks) auch damit ein durchaus akzeptables
Ergebnis erzielen. Für alle, die so denken, finden sich in
Kap.~\ref{chap:Word} einige spezielle Hinweise zum Arbeiten mit
\emph{Word}. Ansonsten sollten für \emph{Word}-Benutzer aber auch
andere Teile dieses Dokuments von Interesse sein, insbesondere die
Abschnitte über Abbildungen und Tabellen
(Kap.~\ref{chap:Abbildungen}) und mathematische Elemente
(Kap.~\ref{chap:Mathematik}).






Übrigens, genau hier am Ende des Einleitungskapitels (und nicht
etwa in der Kurzfassung) ist der richtige Platz, um die
inhaltliche Gliederung der nachfolgenden Arbeit zu beschreiben.
Hier soll dargestellt werden, welche Teile (Kapitel) der Arbeit
welche Funktion haben und wie sie inhaltlich zusammenhängen. Auch
die Inhalte des \emph{Anhangs} -- sofern vorgesehen -- sollten hier
kurz beschrieben werden.

