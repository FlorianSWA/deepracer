\chapter{Introduction}\label{cha:Introduction}
Author: Florian Schwarzl

The goal of this document is to provide a solid foundation for others wanting to work with the AWS DeepRacer. All aspects from starting a training environment to participating on online races are covered. In addition to that this document provides more in depth information about other possible applications of the vehicle and its limitations.


Machine learning has been used extensively in robotics, mainly in the form of image and pattern recognition. This allows robots to follow predefined paths or drive to certain areas. While these forms of machine learning have been in use in our robotics lab for some time, the approach which is provided by reinforcement learning has not seen any application. This was mainly due to a lack of fitting equipment and usable software. The recently acquired AWS DeepRacer offer an opportunity to explore the options and applications of reinforcement learning. As there is currently no suitable environment to put them to good use, we were tasked with creating a foundation for others to learn about reinforcement learning. 

Machine learning is without a doubt an interesting and ever more important aspect of computer science. With a steady increase in both popularity and fields of application, it has risen form a niche spot to a mainstream field. It is therefor only logical to offer students a way to experiments with learning algorithms and autonomous robots. However, it is often costly to acquire the necessary gear for such projects, especially if testing on a real track is intended. Another challenge is the sheer complexity and steep learning curve. Most entry-level machine learning topics rely solely on supervised learning, image recognition being a prime example. This does not mean that it is not suited for beginners, quite the opposite. However, recognising images can be considered dull. Self-driving cars on the other hand show the possibilities of machine learning in an exciting and practical way. Autonomous driving in turn is ill suited for beginners. Amazon aims to fill this void with their own implementation of an reinforcement learning agent, which they call the DeepRacer. The name stems form deep neural network, a component used during training.