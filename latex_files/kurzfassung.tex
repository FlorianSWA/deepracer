\chapter{Kurzfassung}
\begin{german}
Diese Diplomarbeit befasst sich mit der Reinforcement Learning Plattform von AWS, dem AWS DeepRacer. Die DeepRacer Plattform richtet sich and Entwickler und soll einen Einblick in verstärkendes Lernen bieten. Hierfür wird ein maßstabsgetreues Fahrzeug verwendet. Dieses Fahrzeug soll selbstständig eine zuvor eintrainierte Strecke so schnell wie möglich abfahren. Sowohl das Fahrzeug als auch die für das Trainieren benötigte Software steht bei Amazon zum Kauf zur Verfügung.

Der DeepRacer basiert auf einer Art von maschinellem Lernen, die sich "reinforcement leaning", zu deutsch bestärkendes Lernen nennt. Dieser Ansatz beruht auf einer Idee von Belohnung und Bestrafung, um das gewünschte Verhalten des Roboters zu erreichen. Über eine vom Benutzer definierte Belohnungsfunktion wird festgelegt, für welche Aktionen der Roboter Belohnung erhält. Allgemein formuliert ist die Belohnung vom aktuellen Zustand der Umgebung abhängig. Durch seine Aktionen verändert der Roboter seine Umgebung und damit auch die erhaltene Belohnung. Sollte die Belohnung nach einer Reihe von Aktionen konstant hoch sein, merkt sich der Roboter dies für spätere Szenarien.

Anders als beispielsweise überwachtes Lernen, welches oft für Bild- und Spracherkennung verwendet wird, gibt es beim reinforcement Learning keine vordefinierte Menge von Lösungen. Das überwachte bzw. geleitete Lernen basiert auf einer Menge von bereits gelösten Beispielen, anhand derer eine verallgemeinerte Lösungsformel erstellt werden soll. Diese Form des Lernens ermöglicht eine simplere Realisierung, ist jedoch ungeeignet für komplexere Aufgaben. 

Die Idee, dass sich ein Roboter selbst beibringt, wie er eine Aufgabe zu erledigen hat, ist nicht neu. Auch die Anwendung auf Fahrzeuge wurde bereits zuvor umgesetzt. Ausschlaggeben für das DeepRacer Konzept ist die einfache Bedienung und das einsteigerfreundlichere Nivea. 
\end{german}
%An dieser Stelle steht eine Zusammenfassung der Arbeit, Umfang
%max.\ 1 Seite. Im Unterschied zu anderen Kapiteln ist die
%Kurzfassung (und das Abstract) üblicherweise nicht in Abschnitte
%und Unterabschnitte gegliedert. 
%Auch Fußnoten sind hier falsch am Platz.
%
%Kurzfassungen werden übrigens häufig -- zusammen mit Autor und Titel
%der Arbeit -- %
%in Literaturdatenbanken aufgenommen. Es ist daher darauf zu
%achten, dass die Information in der Kurzfassung für sich 
%\emph{allein} (\dah ohne weitere Teile der Arbeit) zusammenhängend und
%abgeschlossen ist. Insbesondere werden an dieser Stelle (wie \ua
%auch im \emph{Titel} der Arbeit und im \emph{Abstract})
%normalerweise \emph{keine Literaturverweise} verwendet! Falls man
%unbedingt solche benötigt -- etwa weil die Arbeit eine
%Weiterentwicklung einer bestimmten, früheren Arbeit darstellt --,
%dann sind \emph{vollständige} Quellenangaben in der Kurzfassung
%selbst notwendig, \zB %
%[\textsc{Zobel} J.: \textit{Writing for Computer Science -- The Art of
%Effective Commu\-nica\-tion}. Springer-Verlag, Singa\-pur, 1997].
%
%Weiters sollte man daran denken, dass bei der Aufnahme in Datenbanken
%Sonderzeichen oder etwa Aufzählungen mit "`Knödellisten"' in der
%Regel verloren gehen. Dasselbe gilt natürlich auch für das 
%\emph{Abstract}.
%
%
%Inhaltlich sollte die Kurzfassung \emph{keine} Auflistung der
%einzelnen Kapitel sein (dafür ist das Einleitungskapitel
%vorgesehen), sondern dem Leser einen kompakten, inhaltlichen
%Überblick über die gesamte Arbeit verschaffen. Der hier verwendete
%Aufbau ist daher zwangsläufig anders als der in der Einleitung.

