\chapter{Abstract}

\begin{english} %switch to English language rules
Author: Sebastian Rohrer \newline
This thesis is about the reinforcement learning platform from AWS, the AWS DeepRacer. The DeepRacer platform is aimed at developers and is intended to provide insight into reinforcement learning. For this purpose, a scaled vehicle is used. This vehicle is supposed to autonomously drive a previously trained route as fast as possible. Both the vehicle and the software needed for training are available for purchase from Amazon.

The DeepRacer is based on a type of machine learning called reinforcement leaning. This approach relies on an idea of reward and punishment to achieve the desired behaviour of the robot. A user-defined reward function is used to determine the actions for which the robot receives reward. Generally speaking, the reward depends on the current state of the environment. Through its actions, the robot changes its environment and thus also the reward it receives. If the reward is consistently high after a series of actions, the robot remembers this for later scenarios.

Unlike, for example, supervised learning, which is often used for image and speech recognition, reinforcement learning does not have a predefined set of solutions. Supervised or guided learning is based on a set of previously solved examples, which are used to create a generalised solution formula. This form of learning allows a simpler realisation, but is unsuitable for more complex tasks. 

The idea of a robot teaching itself how to perform a task is not new. It has also been applied to vehicles before. The deciding factor for the DeepRacer concept is the simple operation and the more beginner-friendly Niveau. 
\end{english}
