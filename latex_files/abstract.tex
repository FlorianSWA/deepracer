\chapter{Abstract}

\begin{english} %switch to English language rules
After our school received two Amazon AWS DeepRacer vehicles, those vehicles had to be put to good use. With this thesis we paved the way for future students to learn about machine learning in an easy and understandable way. More experienced students will be more interested in the ability to change the training algorithm and other parameters, thanks to the possibility of local training. 

This document begins with an explanation of reinforcement learning and machine learning in general. The second chapter covers theoretical principles behind the DeepRacer implementation. Understanding these is, while not essential to working with the platform, this knowledge will help understanding the implementation described later on. The third chapter explains how the DeepRacer platform implements reinforcement learning. It covers the concept and all theoretical aspects of the learning process. The DeepRacer implementation of reinforcement learning differs from others in many ways, most notably are the limitations put on the training simulation in order to create an even ground for competitive races. After that follows an overview of the car and its parts, how it handles and what else can be done with it. The fifth chapter consists of a summary of the building process of the physical track.

\end{english}
