\chapter{Conclusion}
Author: Sebastian Rohrer \newline
In the context of this diploma thesis, it was researched whether the Amazon Deepracer represents a use-case for the Robo4You robotics association \repeatfootcite{Robo4you}, and whether this is worthwhile to acquire machine learning skills. The goal was to create a reference point, including documentation and code templates, to use the Deepracer and train it locally. We have found that the race track available to us is insufficient to train a consistent DeepRacer because it has too high deviations from the given one. The car, while driving, detects reflections on the ground, which it mistakes for edge lines and then drives uncontrollably. Likewise, the green areas at the edge of the track could not be implemented on the test track, increasing the probability of errors. Besides, since other people also use the lab,a strong wear due to friction was experienced, which was also a significant error source in the test rounds. Thus, an attempt was made to redesign the track that is trained to the model to create a more realistic environment. This did not work because AWS restricts the developers access to the track's core files, which describe the shape and representation. These are downloaded from an Amazon server at the beginning of the training. Changes in the engines' calibration have had positive effects, but they were not large enough to compensate for the inadequately built racetrack. There have also been setbacks, as Amazon only profits on cloud-based usage, there are always updates that make local training more difficult and complicated. Unfortunately, there is no documentation or information on the Internet about the changes made during an update, so a lot of debugging and troubleshooting must be done. For this reason, help pages such as the DeepRacer Wiki\repeatfootcite{Deepracer-Wiki} have been created by users on the Internet. On such community-driven web pages a lot of information can be gathered, but these pages are primarily outdated and do not offer solutions for the actual problems. In general, the experience and skills gained through DeepRacer about machine learning and robotics are great. The benefit for the robotics club can be seen in the acquisition of machine learning skills, as well as the introduction to cloud usage. Also, a competitive environment can be created to participate in competitions with the prospect of the world championship, thereby strengthening programming and algorithmic skills. However, this can only be done using a track-tuned for DeepRacer according to Amazon's specifications, as on more major championships, cars have to compete against each other on accurate tracks. This project has shown that even if machine learning algorithms work well in simulation, this is not 1:1 transferable to real-life because even with the slightest inconsistencies, the robot must take fatal consequences.
\section{Practibality}
In the next section, different ways of continuing this project are described, as the DeepRacer has many ways of being useful. 
\subsection{Continuation}
Building on this research project, the car can be trained competitively and used to participate in competitions. This is possible under the condition of new track construction and good programming skills. \newline
Since the DeepRacer is generally easy to handle and uncomplicated to operate, it is also possible to use the car to introduce young roboticists to the world of machine learning and provide a ramp for further projects.
\subsection{Other Usecases}
Since the car as a whole is only a robot with motors and a camera, it can be used as an autonomous vehicle for other projects. This lends itself well because it has a Raspberry Pi 3, which runs the operating system Ubuntu. The DeepRacer works with the Robot Operating System (ROS), which offers many different configuration and expansion options. Furthermore, the car's REST interface can be used to guarantee remote control or other interactive functionalities.