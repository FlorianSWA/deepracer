\chapter{Thesis}
\label{cha:Diplomschrift}

\section{The Vehicle}

The DeepRacer itself represents a core part of this thesis, as it is used to test the trained models in a real-world environment. Our school acquired two of these cars. one of which we are using for this thesis. Below is a list of all parts included with one DeepRacer vehicle.

\begin{enumerate}
    \item Vehicle chassis
    \item Vehicle body cover
    \item Compute module battery
    \item Power cable and power adapter
    \item Vehicle battery
\end{enumerate}

\subsection{Chassis and Accessories}
The vehicle itself consists primarily of a four wheel drive chassis which holds all other components. The chassis can be further separated into a lower part, which contains the brushed electric motor, and an upper part that carries the compute module and a power bank to supply it. The entire car is build on a scale of 1/18\footnote{Source: https://docs.aws.amazon.com/deepracer/latest/developerguide/deepracer-prep-vehicle.html} to a real car, meaning proportions like distance between wheels were kept realistic. This is especially apparent while driving, as one would expect better manoeuvrability and smaller turning radius.
At the front there are three USB ports used to mount the camera and other equipment like keyboards. As a crucial part of any self-driving car, the camera provided with the DeepRacer provides a 4-megapixel image directly to the compute module. Since we are working with the first edition of the DeepRacer vehicle and not with the newer DeepRacer Evo, which has stereo cameras and a LiDAR sensor, object avoidance and head-to-head racing are not supported by default. It is however possible to purchase an upgrade kit for 249,00 \$, which includes an additional stereo camera and the LiDAR sensor.

\subsection{Compute Module}
These ports are also used to connect a mouse and keyboard in order to access the compute module. This module is most likely a Raspberry Pi running Ubuntu Linux. It also offers a HDMI port so that you are able to connect a monitor and

\section{Machine Learning and Autonomous Driving}


 \section{Local Training}
 
 One of the major drawbacks of using the DeepRacer in a learning environment are the costs of training. Amazon offers offers easy, albeit functionally limited ways of training RL models in their cloud services. This sort of contradicts the intended use, as the DeepRacer is supposed to offer a simple and affordable entry into the ways of machine learning. Below is the pricing table. \footnote{Cited date 2020/08/20}
 \begin{table}
 \caption{Pricing for model training with AWS services.}
 \label{tab:services}
 \centering
 \setlength{\tabcolsep}{5mm}
 \def\arraystretch{1.25}
 \begin{tabular}{|r|r|c|c|c|}
 \emph{service} & \emph{pricing} \\
 \hline\hline
 training and evaluation & 3.50 \$ per hour \\
 \hline
 model storage & 0.023 \$ per GB \\
 \hline
 \end{tabular}
 \end{table}
 In order to circumvent this cost barrier we -- like others before us -- began setting up a training environment on one of the more powerful computers in the robotics lab.
 The setup for local training is available on GitHub \footcite{https://github.com/aws-deepracer-community/deepracer}. In order to function properly the computer had to meet the following requirements:
 \begin{itemize}
 \item A Linux distribution, preferably Ubuntu
 \item NVIDIA grafic processor and proper dirvers
 \item Docker
 \item Python
 \item Minio, a S3 simulator
 \end{itemize}