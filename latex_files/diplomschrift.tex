\chapter{Local Training}
\label{cha:Diplomschrift}
 
 One of the major drawbacks of using the DeepRacer in a learning environment are the costs of training. Amazon offers easy, albeit functionally limited ways of training RL models in their cloud services. This sort of contradicts the intended use, as the DeepRacer is supposed to offer a simple and affordable entry into the ways of machine learning. Below is the pricing table. \footnote{Cited date 2020/08/20}
 \begin{table}
 \caption{Pricing for model training with AWS services.}
 \label{tab:services}
 \centering
 \setlength{\tabcolsep}{5mm}
 \def\arraystretch{1.25}
 \begin{tabular}{|r|r|c|c|}
 \hline
 \textbf{service} & \textbf{pricing} \\
 \hline\hline
 training and evaluation & 3.50 US\$ per hour \\
 \hline
 model storage & 0.023 US\$ per GB \\
 \hline
 \end{tabular}
 \end{table}
 In order to circumvent this cost barrier we -- like others before us -- began setting up a training environment on one of the more powerful computers in the robotics lab.
 The setup for local training is available on GitHub \footcite{https://github.com/aws-deepracer-community/deepracer}. In order to function properly the computer had to meet the following requirements:
 \begin{itemize}
 \item A Linux distribution, preferably Ubuntu
 \item NVIDIA grafic processor and proper dirvers
 \item Docker
 \item Python
 \item Minio, a S3 simulator
 \end{itemize}